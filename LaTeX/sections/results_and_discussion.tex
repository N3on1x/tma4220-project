% !TEX root = ../main/main.tex
\section{Results and discussion}
 
 \begin{figure}[h]
 \begin{subfigure}[t] {0.23\textwidth}
 \centering
 \includegraphics[width=0.7\textwidth]{../figures/heatsink4_h105_gmf005.png}
 \caption{"Placeholder"}
 \label{fig:res_4_1}
 \end{subfigure}
 ~
  \begin{subfigure}[t] {0.23\textwidth}
 \centering
 \includegraphics[width=0.7\textwidth]{../figures/heatsink8_h105_gmf005.png}
 \caption{"Placeholder"}
 \label{fig:res_8_1}
 \end{subfigure}
 ~
 \begin{subfigure}[t] {0.23\textwidth}
 \centering
 \includegraphics[width=0.7\textwidth]{../figures/heatsink4_h205_gmf005.png}
 \caption{"Placeholder"}
 \label{fig:res_4_2}
 \end{subfigure}
 ~
 \begin{subfigure}[t] {0.23\textwidth}
 \centering
 \includegraphics[width=0.7\textwidth]{../figures/heatsink8_h205_gmf005.png}
 \caption{"Placeholder"}
 \label{fig:res_8_2}
 \end{subfigure}
 \caption{These are the different meshes we tried}
 \label{fig:mesh_temps}
 \end{figure}
 
\subsection{Visualization of results}
Visualizing results from simulations is important in order to understand the result. when having a function in three dimensions it is necessary to make choices of what part of the result which should be displayed. The natural way of doing this is either with iso-surfaces or cutting-planes. The geometries in \ref{fig:meshes} are somewhat symmetric, especially along the fins, which suggests that the temperature field should be more or less constant along this axis. In 3D visualizing software this is easier to see, and in our solution this is indeed the case. Therefore figure \ref{fig:mesh_temps} represents a cutting plane normal to the find, which should give a good representation of the temperature field in whole geometry.

\subsection{Assumptions and Physical Interpretation}
In our analysis we have made some assumptions and initial choices for the boundary conditions to simplify
