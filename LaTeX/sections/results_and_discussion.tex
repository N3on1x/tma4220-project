% !TEX root = ../main/main.tex
\section{Results and discussion}
 
 \begin{figure}[h]
 \begin{subfigure}[t] {0.23\textwidth}
 \centering
 \includegraphics[width=0.7\textwidth]{../figures/heatsink4_h105_gmf005.png}
 \caption{4 fins, 1 height}
 \label{fig:mesh_temps_res_4_1}
 \end{subfigure}
 ~
  \begin{subfigure}[t] {0.23\textwidth}
 \centering
 \includegraphics[width=0.7\textwidth]{../figures/heatsink8_h105_gmf005.png}
 \caption{8 fins, 1 height}
 \label{fig:mesh_temps_res_8_1}
 \end{subfigure}
 ~
 \begin{subfigure}[t] {0.23\textwidth}
 \centering
 \includegraphics[width=0.7\textwidth]{../figures/heatsink4_h205_gmf005.png}
 \caption{4 fins, 2 height}
 \label{fig:mesh_temps_res_4_2}
 \end{subfigure}
 ~
 \begin{subfigure}[t] {0.23\textwidth}
 \centering
 \includegraphics[width=0.7\textwidth]{../figures/heatsink8_h205_gmf005.png}
 \caption{8 fins, 2 height}
 \label{fig:mesh_temps_res_8_2}
 \end{subfigure}
 \caption{Temperatures in the middle of the heat sink for the different meshes.}
 \label{fig:mesh_temps}
 \end{figure}
 
\subsection{Visualization of results}
Visualizing results from simulations is important in order to understand the result. when having a function in three dimensions it is necessary to make choices of what part of the result which should be displayed. The natural way of doing this is either with iso-surfaces or cutting-planes. The geometries in \ref{fig:meshes} are somewhat symmetric, especially along the fins, which suggests that the temperature field should be more or less constant along this axis. In 3D visualizing software this is easier to see, and in our solution this is indeed the case. Therefore figure \ref{fig:mesh_temps} represents a cutting plane normal to the find, which should give a good representation of the temperature field in whole geometry.

\subsection{Assumptions and Physical Interpretation}
In our analysis we have made some assumptions and initial choices for the boundary conditions to make a somewhat simplified mathematical model. For the temperature of the bottom of the heat sink base we enforced a constant Dirichlet boundary condition of \SI{80}{\celsius}. This choice was made because it is an estimate of the temperature of a very warm processor working at maximum capacity. By enforcing Dirichlet boundary condition we disallow the hypothetical processor to actually cool down, which is not necessarily a realistic assumption. We would suggest that some kind of Neumann boundary condition would be better suited if you want to see a cooling effect. Another modeling choice made regarding the Robin conditions on the rest of the boundary is that the ambient temperature in the description of the boundary value problem \eqref{eq:heat_bvp} is also constant. This means that even in the tight spaces in between the fins the air temperature is constant. You could think of it as an average temperature on the boundary. One could maybe improve on this assumption by enforcing different ambient temperature on the outer boundary and the boundary in between the fins. Besides you would probably need a very powerful fan to transport the heat away to keep the ambient temperature at 20 degrees as assumed here.
If there were many more fins and they were a lot thinner and closer together it would be a good point to take into account the heat radiation from one fin to another. Also there would be challenges regarding the mesh. Using tetrahedron elements one could expect the elements to become very irregular. A solution could be to use brick prism elements, or alternatively a 2D mesh of the flat fins.

In figure \ref{fig:mesh_temps} we can see the temperatures for the different meshes. The temperature ranges from about \SI{59}{\celsius} to \SI{80}{\celsius} for figure \ref{fig:mesh_temps_res_4_1} and \ref{fig:mesh_temps_res_8_1}, and from \SI{39}{\celsius} to \SI{80}{\celsius} for figure \ref{fig:mesh_temps_res_4_2} and \ref{fig:mesh_temps_res_8_2}. The temperature profiles along the fins are virtually identical between fins of the same height. The Dirichlet boundary works like a thermal reservoir.